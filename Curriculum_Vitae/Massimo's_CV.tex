\documentclass[10pt]{article}
\usepackage[utf8]{inputenc}
\usepackage[b4paper]{geometry}
\geometry{margin=0.5in}
\usepackage{enumitem}
\usepackage{hyperref}
\renewcommand{\familydefault}{\sfdefault}
\setlength{\parindent}{0pt}
\setlength{\parskip}{6pt}

\begin{document}

% Name and Contact Information
\begin{center}
    {\Huge \textbf{Massimo Caruso}} \\
    \vspace{1mm}
    {\small (514) 944-5977 $\vert$ massimo02caruso@gmail.com $\vert$ \href{https://linkedin.com/in/massimocaruso}{linkedin.com/in/massimocaruso} $\vert$ \href{https://github.com/Extinctable}{github.com/Extinctable}}
\end{center}

\vspace{-8mm}

% Education Section
\section*{Education}
\vspace{-2mm}
\hrule
\vspace{0mm}

\textbf{Montreal, Canada} \hfill \textbf{Concordia University} \hfill \textbf{Jan 2023 -- Present} 
\vspace{-4mm}
\begin{itemize}[left=0.15in, itemsep=0pt]
    \item \textbf{Major:} Software Engineering, BEng
\end{itemize}

% Experience Section
\section*{Experience}
\vspace{-2mm}
\hrule
\vspace{0mm}

\textbf{CTF Challenge Designer, } \hfill \textbf{AtHackCTF} \hfill \textbf{Nov 2024 -- Present} \\
{Hackathon {\href{https://www.athackctf.com}{(www.athackctf.com)}: Capture the flag based hackathon.}}
\vspace{-4mm}
\begin{itemize}[left=0.15in, itemsep=0pt]
    \item Designed and implemented RFID-based Capture the Flag (CTF) challenges that simulate real-world security scenarios, providing participants with engaging and educational problem-solving experiences.
    \item Created challenges involving RFID technologies, leveraging knowledge of protocols, hardware (e.g., RFID readers and tags), and software tools to develop realistic attack and defense simulations.
    \item Focused on balancing difficulty and educational value, ensuring participants gain practical skills in RFID security, including vulnerability identification and exploitation.
\end{itemize}

% Projects Section
\section*{Projects}
\vspace{-2mm}
\hrule
\vspace{0mm}

\textbf{Food Database System}
\vspace{-4mm}
\begin{itemize}
    \item Developed an online nutritional database integrating structured data from two APIs (FatSecret and TheMealDB) to provide detailed nutritional information, recipes, and dietary metadata.
    \item Scraped, cleaned, and validated JSON data from APIs using Python scripts, addressing null and duplicate values. Data was stored in relational (PostgreSQL) and non-relational (MongoDB) databases to leverage hybrid data management.
    \item Designed an ER model with 15 interconnected tables for PostgreSQL and collections for MongoDB ("recipes," "directions," and "foods"). Optimized queries for both SQL and NoSQL, including indexing and aggregate operations to enhance performance.
    \item Automated migration of structured data from PostgreSQL to MongoDB using Python scripts, ensuring seamless linkage of primary and foreign keys during the process.
    \item Resolved inefficiencies in FatSecret API data retrieval by implementing a random recipe name generator to bypass large ID gaps, significantly improving data extraction speed and accuracy.
\end{itemize}

\textbf{Teacher-Student Feedback Web Application} 
\vspace{-4mm}
\begin{itemize}[left=0.15in, itemsep=0pt]
    \item Designed and implemented responsive frontend components, including an interactive sidebar, header, and landing page, ensuring smooth navigation and adaptability across devices.
    \item Developed and connected the frontend for the feedback and contact pages to backend APIs, allowing users to view feedback and submit messages with real-time confirmations.
    \item Designed and hosted relational databases on Microsoft Azure, optimizing SQL queries for efficient data retrieval and manipulation.
    \item Conducted acceptance tests, resolved bugs, and ensured feature functionality for a seamless user experience across various devices.
    \item Managed the repository with version control, reviewed pull requests, implemented coding standards, and documented processes to ensure project quality and organization.
\end{itemize}

\textbf{Linear Regression Model} 
\vspace{-4mm}
\begin{itemize}[left=0.15in, itemsep=0pt]
    \item Developed a multiple linear regression model to analyze factors influencing life expectancy using WHO data from 193 countries (2000–2015).
    \item Cleaned and preprocessed data by managing missing values, removing outliers, and converting categorical variables into quantitative data using Python.
    \item Applied backward elimination to reduce predictors from 20 to 6, addressing multicollinearity and improving model accuracy (adjusted R² = 0.771).
    \item Used Python (Pandas, NumPy, Scikit-learn) for analysis, hypothesis testing, and modeling; visualized correlations using heatmaps and scatter plots.
    \item Predicted 2013 Canada life expectancy within 0.5 years of WHO’s reported value, demonstrating model reliability and efficiency.
\end{itemize}

% Skills Section
\section*{Skills}
\vspace{-2mm}
\hrule
\vspace{0mm}
\textbf{Languages:} (Proficient): Java, HTML/CSS, Javascript, Python, SQL; (Familiar): C, C++, Erlang \\
\textbf{Frameworks:} React, Node.js, Flask  \\
\textbf{Libraries:} Pandas, NumPy, Matplotlib, Scikit-learn \\
\textbf{Developer Tools:} Git, MongoDB, PostgreSQL, Neo4j, VS Code, Eclipse, Jupyter Notebook \\
\textbf{Relevant Courses:} Data Structures and Algorithms, Operating Systems, System Hardware, OOP, Databases

\end{document}
