\documentclass[10pt]{article}
\usepackage[utf8]{inputenc}
\usepackage[b4paper]{geometry}
\geometry{margin=0.5in}
\usepackage{enumitem}
\usepackage[hidelinks]{hyperref}
\usepackage{fontawesome}
\renewcommand{\familydefault}{\sfdefault}
\setlength{\parindent}{0pt}
\setlength{\parskip}{6pt}

\begin{document}

% Commandes pour mes informations de contact
\newcommand{\fullname}{Massimo Caruso}
\newcommand{\phone}{(514) 944-5977}
\newcommand{\email}{massimo02caruso@gmail.com}
\newcommand{\linkedin}{linkedin.com/in/massimocaruso}
\newcommand{\github}{github.com/Extinctable}

% Commandes pour les informations du destinataire
\newcommand*{\hiringManager}{Annie Theoret}
\newcommand*{\company}{NEOPOS}
\newcommand*{\address}{8150 Ave Marco-Polo, Montréal, Québec H1E 5Y7}

% Commandes pour les paragraphes du corps
\newcommand*{\position}{stagiaire en génie logiciel}
\newcommand*{\companyDetails}{son accent sur des solutions innovantes, fiables et efficaces de systèmes POS qui permettent aux entreprises d'optimiser leurs opérations et d'améliorer l'expérience client}
\newcommand*{\positionDetails}{contribuer au développement de logiciels de pointe pour les systèmes POS, en utilisant mes compétences pour créer des solutions évolutives et conviviales}

% Nom et informations de contact
\begin{center}
    {\Huge \textbf{\fullname}} \\
    \vspace{1mm}
    {\small 
    \faPhone \hspace{0mm} \href{tel:+15149445977}{\phone} $\vert$ 
    \faEnvelope \hspace{0mm} \href{mailto:massimo02caruso@gmail.com}{\email} $\vert$ 
    \faLinkedin \hspace{0mm} \href{https://linkedin.com/in/massimocaruso}{\linkedin} $\vert$ 
    \faGithub \hspace{0mm} \href{https://github.com/Extinctable}{\github}
    }
\end{center}

% Informations du destinataire
\vspace{1.5em}
\hiringManager \hfill \today \\
\company \\
\address

% Ouverture
\vspace{1.5em}
Chère \hiringManager,

Je suis ravi de postuler au poste de \position \space chez \company. En tant qu'étudiant en troisième année de génie logiciel à l'Université Concordia, je possède une solide base en développement logiciel, ainsi qu'une expérience pratique acquise au cours de projets académiques et professionnels.

% Corps
Pendant mon mandat en tant que concepteur de défis CTF chez \href{https://www.athackctf.com}{AtHackCTF}, j'ai développé des défis RFID Capture the Flag pour simuler des scénarios de sécurité réels. Cette expérience a renforcé mes compétences techniques en analyse de systèmes, en intégration matériel/logiciel, et en conception d'environnements interactifs pour la résolution de problèmes, des compétences qui correspondent directement aux exigences du rôle de \position.

De plus, mes projets académiques, tels que la création d'un système de base de données alimentaire en ligne et une application de retour d'information entre enseignants et étudiants, m'ont permis d'affiner mon expertise en Python, JavaScript et SQL, tout en travaillant avec des API et des bases de données relationnelles. Ces projets m'ont appris à gérer le développement logiciel de bout en bout, optimiser les performances du système et me concentrer sur la conception centrée sur l'utilisateur.

% Pourquoi cette entreprise
Ce qui m'enthousiasme le plus chez \company, c'est \companyDetails. Je suis particulièrement attiré par l'opportunité de \positionDetails.

% Conclusion
Je suis convaincu que mes compétences techniques, mes réalisations académiques et ma passion pour le génie logiciel font de moi un candidat solide pour le poste de \position. Je serais ravi de discuter de la manière dont mes expériences répondent aux besoins de votre équipe. N'hésitez pas à me contacter au \href{tel:+15149445977}{\phone} ou à \href{mailto:\email}{\email}.

\vspace{1.5em}
Je vous remercie de considérer ma candidature. J'espère avoir l'opportunité de contribuer au succès de \company.

% Signature
\vspace{2em}
Cordialement, \\
\fullname

\newpage

% Nom et informations de contact
\begin{center}
    {\Huge \textbf{\fullname}} \\
    \vspace{1mm}
    {\small 
    \faPhone \hspace{0mm} \href{tel:+15149445977}{\phone} $\vert$ 
    \faEnvelope \hspace{0mm} \href{mailto:massimo02caruso@gmail.com}{\email} $\vert$ 
    \faLinkedin \hspace{0mm} \href{https://linkedin.com/in/massimocaruso}{\linkedin} $\vert$ 
    \faGithub \hspace{0mm} \href{https://github.com/Extinctable}{\github}
    }
\end{center}

\vspace{-8mm}

% Section Éducation
\section*{Éducation}
\vspace{-2mm}
\hrule
\vspace{0mm}

\textbf{Montréal, Canada} \hfill \textbf{Université Concordia} \hfill \textbf{Janvier 2023 -- Présent} 
\vspace{-4mm}
\begin{itemize}[left=0.15in, itemsep=0pt]
    \item \textbf{Spécialisation :} Génie logiciel, BEng
\end{itemize}

% Section Expérience
\section*{Expérience}
\vspace{-2mm}
\hrule
\vspace{0mm}

\textbf{Concepteur de défis CTF, } \hfill \textbf{AtHackCTF} \hfill \textbf{Novembre 2024 -- Présent} \\
{Hackathon {\href{https://www.athackctf.com}{(www.athackctf.com)} : Hackathon basé sur le concept Capture the Flag.}}
\vspace{-4mm}
\begin{itemize}[left=0.15in, itemsep=0pt]
    \item Conception et implémentation de défis RFID Capture the Flag (CTF) simulant des scénarios de sécurité réels, offrant aux participants des expériences éducatives et engageantes.
    \item Création de défis utilisant des technologies RFID, en exploitant les protocoles, matériels (lecteurs et étiquettes RFID) et outils logiciels pour développer des simulations réalistes d'attaques et de défenses.
    \item Équilibrage entre difficulté et valeur éducative, garantissant que les participants acquièrent des compétences pratiques en sécurité RFID, y compris l'identification et l'exploitation des vulnérabilités.
\end{itemize}

% Section Projets
\section*{Projets}
\vspace{-2mm}
\hrule
\vspace{0mm}

\textbf{Système de base de données alimentaire}
\vspace{-4mm}
\begin{itemize}[left=0.15in, itemsep=0pt, label=--]
    \item Développement d'une base de données nutritionnelle en ligne intégrant des données structurées de deux APIs (FatSecret et TheMealDB) pour fournir des informations nutritionnelles détaillées, des recettes et des métadonnées diététiques.
    \item Nettoyage, validation et extraction des données JSON depuis les APIs à l'aide de scripts Python, en traitant les valeurs nulles et les doublons. Les données ont été stockées dans des bases relationnelles (PostgreSQL) et non relationnelles (MongoDB) pour tirer parti d'une gestion hybride.
    \item Conception d'un modèle ER avec 15 tables interconnectées pour PostgreSQL et des collections pour MongoDB ("recettes", "instructions" et "aliments"). Optimisation des requêtes SQL et NoSQL grâce à des index et des opérations d'agrégation pour améliorer les performances.
    \item Migration automatisée des données structurées de PostgreSQL vers MongoDB à l'aide de scripts Python, assurant un lien fluide entre les clés primaires et étrangères.
    \item Résolution des inefficacités dans la récupération des données de l'API FatSecret en implémentant un générateur de noms de recettes aléatoires pour contourner les grands écarts d'ID, améliorant considérablement la vitesse et la précision de l'extraction des données.
\end{itemize}

\textbf{Application de retour d'information entre enseignants et étudiants} 
\vspace{-4mm}
\begin{itemize}[left=0.15in, itemsep=0pt, label=--]
    \item Conception et implémentation de composants front-end responsifs, y compris une barre latérale interactive, un en-tête et une page d'accueil, garantissant une navigation fluide et une adaptabilité sur tous les appareils.
    \item Développement et connexion du front-end pour les pages de feedback et de contact aux APIs backend, permettant aux utilisateurs de consulter les retours et de soumettre des messages avec des confirmations en temps réel.
    \item Conception et hébergement de bases de données relationnelles sur Microsoft Azure, optimisant les requêtes SQL pour une récupération et une manipulation efficaces des données.
    \item Réalisation de tests d'acceptation, résolution des bogues et vérification de la fonctionnalité des fonctionnalités pour une expérience utilisateur fluide sur divers appareils.
    \item Gestion du dépôt avec le contrôle de version, révision des pull requests, mise en œuvre de normes de codage et documentation des processus pour assurer la qualité et l'organisation du projet.
\end{itemize}

\textbf{Modèle de régression linéaire} 
\vspace{-4mm}
\begin{itemize}[left=0.15in, itemsep=0pt, label=--]
    \item Développement d'un modèle de régression linéaire multiple pour analyser les facteurs influençant l'espérance de vie à l'aide des données de l'OMS couvrant 193 pays (2000–2015).
    \item Nettoyage et prétraitement des données en gérant les valeurs manquantes, en supprimant les valeurs aberrantes et en convertissant les variables catégoriques en données quantitatives à l'aide de Python.
    \item Application de l'élimination arrière pour réduire les prédicteurs de 20 à 6, en traitant la multicolinéarité et en améliorant la précision du modèle (R² ajusté = 0,771).
    \item Utilisation de Python (Pandas, NumPy, Scikit-learn) pour l'analyse, les tests d'hypothèses et la modélisation ; visualisation des corrélations à l'aide de cartes thermiques et de diagrammes de dispersion.
    \item Prédiction de l'espérance de vie du Canada en 2013 avec une précision de 0,5 an par rapport à la valeur rapportée par l'OMS, démontrant la fiabilité et l'efficacité du modèle.
\end{itemize}

% Section Compétences
\section*{Compétences}
\vspace{-2mm}
\hrule
\vspace{0mm}
\textbf{Langages :} (Maîtrisé) : Java, LaTeX, HTML/CSS, Javascript, Python, SQL ; (Connaissances) : C, C++, Clojure, Erlang \\
\textbf{Frameworks :} React, Node.js, Flask, Express.js  \\
\textbf{Bibliothèques :} Pandas, NumPy, Matplotlib, Scikit-learn \\
\textbf{Outils de développement :} Git, Docker, MongoDB, PostgreSQL, Neo4j, VS Code, Eclipse, Jupyter Notebook \\
\textbf{Cours pertinents :} Structures de données et algorithmes, systèmes d'exploitation, matériel informatique, POO, bases de données

\end{document}

